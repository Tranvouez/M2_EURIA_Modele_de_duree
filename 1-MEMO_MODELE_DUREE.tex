% Options for packages loaded elsewhere
\PassOptionsToPackage{unicode}{hyperref}
\PassOptionsToPackage{hyphens}{url}
%
\documentclass[
]{article}
\usepackage{amsmath,amssymb}
\usepackage{iftex}
\ifPDFTeX
  \usepackage[T1]{fontenc}
  \usepackage[utf8]{inputenc}
  \usepackage{textcomp} % provide euro and other symbols
\else % if luatex or xetex
  \usepackage{unicode-math} % this also loads fontspec
  \defaultfontfeatures{Scale=MatchLowercase}
  \defaultfontfeatures[\rmfamily]{Ligatures=TeX,Scale=1}
\fi
\usepackage{lmodern}
\ifPDFTeX\else
  % xetex/luatex font selection
\fi
% Use upquote if available, for straight quotes in verbatim environments
\IfFileExists{upquote.sty}{\usepackage{upquote}}{}
\IfFileExists{microtype.sty}{% use microtype if available
  \usepackage[]{microtype}
  \UseMicrotypeSet[protrusion]{basicmath} % disable protrusion for tt fonts
}{}
\makeatletter
\@ifundefined{KOMAClassName}{% if non-KOMA class
  \IfFileExists{parskip.sty}{%
    \usepackage{parskip}
  }{% else
    \setlength{\parindent}{0pt}
    \setlength{\parskip}{6pt plus 2pt minus 1pt}}
}{% if KOMA class
  \KOMAoptions{parskip=half}}
\makeatother
\usepackage{xcolor}
\usepackage[margin=1in]{geometry}
\usepackage{color}
\usepackage{fancyvrb}
\newcommand{\VerbBar}{|}
\newcommand{\VERB}{\Verb[commandchars=\\\{\}]}
\DefineVerbatimEnvironment{Highlighting}{Verbatim}{commandchars=\\\{\}}
% Add ',fontsize=\small' for more characters per line
\usepackage{framed}
\definecolor{shadecolor}{RGB}{248,248,248}
\newenvironment{Shaded}{\begin{snugshade}}{\end{snugshade}}
\newcommand{\AlertTok}[1]{\textcolor[rgb]{0.94,0.16,0.16}{#1}}
\newcommand{\AnnotationTok}[1]{\textcolor[rgb]{0.56,0.35,0.01}{\textbf{\textit{#1}}}}
\newcommand{\AttributeTok}[1]{\textcolor[rgb]{0.13,0.29,0.53}{#1}}
\newcommand{\BaseNTok}[1]{\textcolor[rgb]{0.00,0.00,0.81}{#1}}
\newcommand{\BuiltInTok}[1]{#1}
\newcommand{\CharTok}[1]{\textcolor[rgb]{0.31,0.60,0.02}{#1}}
\newcommand{\CommentTok}[1]{\textcolor[rgb]{0.56,0.35,0.01}{\textit{#1}}}
\newcommand{\CommentVarTok}[1]{\textcolor[rgb]{0.56,0.35,0.01}{\textbf{\textit{#1}}}}
\newcommand{\ConstantTok}[1]{\textcolor[rgb]{0.56,0.35,0.01}{#1}}
\newcommand{\ControlFlowTok}[1]{\textcolor[rgb]{0.13,0.29,0.53}{\textbf{#1}}}
\newcommand{\DataTypeTok}[1]{\textcolor[rgb]{0.13,0.29,0.53}{#1}}
\newcommand{\DecValTok}[1]{\textcolor[rgb]{0.00,0.00,0.81}{#1}}
\newcommand{\DocumentationTok}[1]{\textcolor[rgb]{0.56,0.35,0.01}{\textbf{\textit{#1}}}}
\newcommand{\ErrorTok}[1]{\textcolor[rgb]{0.64,0.00,0.00}{\textbf{#1}}}
\newcommand{\ExtensionTok}[1]{#1}
\newcommand{\FloatTok}[1]{\textcolor[rgb]{0.00,0.00,0.81}{#1}}
\newcommand{\FunctionTok}[1]{\textcolor[rgb]{0.13,0.29,0.53}{\textbf{#1}}}
\newcommand{\ImportTok}[1]{#1}
\newcommand{\InformationTok}[1]{\textcolor[rgb]{0.56,0.35,0.01}{\textbf{\textit{#1}}}}
\newcommand{\KeywordTok}[1]{\textcolor[rgb]{0.13,0.29,0.53}{\textbf{#1}}}
\newcommand{\NormalTok}[1]{#1}
\newcommand{\OperatorTok}[1]{\textcolor[rgb]{0.81,0.36,0.00}{\textbf{#1}}}
\newcommand{\OtherTok}[1]{\textcolor[rgb]{0.56,0.35,0.01}{#1}}
\newcommand{\PreprocessorTok}[1]{\textcolor[rgb]{0.56,0.35,0.01}{\textit{#1}}}
\newcommand{\RegionMarkerTok}[1]{#1}
\newcommand{\SpecialCharTok}[1]{\textcolor[rgb]{0.81,0.36,0.00}{\textbf{#1}}}
\newcommand{\SpecialStringTok}[1]{\textcolor[rgb]{0.31,0.60,0.02}{#1}}
\newcommand{\StringTok}[1]{\textcolor[rgb]{0.31,0.60,0.02}{#1}}
\newcommand{\VariableTok}[1]{\textcolor[rgb]{0.00,0.00,0.00}{#1}}
\newcommand{\VerbatimStringTok}[1]{\textcolor[rgb]{0.31,0.60,0.02}{#1}}
\newcommand{\WarningTok}[1]{\textcolor[rgb]{0.56,0.35,0.01}{\textbf{\textit{#1}}}}
\usepackage{graphicx}
\makeatletter
\def\maxwidth{\ifdim\Gin@nat@width>\linewidth\linewidth\else\Gin@nat@width\fi}
\def\maxheight{\ifdim\Gin@nat@height>\textheight\textheight\else\Gin@nat@height\fi}
\makeatother
% Scale images if necessary, so that they will not overflow the page
% margins by default, and it is still possible to overwrite the defaults
% using explicit options in \includegraphics[width, height, ...]{}
\setkeys{Gin}{width=\maxwidth,height=\maxheight,keepaspectratio}
% Set default figure placement to htbp
\makeatletter
\def\fps@figure{htbp}
\makeatother
\setlength{\emergencystretch}{3em} % prevent overfull lines
\providecommand{\tightlist}{%
  \setlength{\itemsep}{0pt}\setlength{\parskip}{0pt}}
\setcounter{secnumdepth}{5}
\ifLuaTeX
  \usepackage{selnolig}  % disable illegal ligatures
\fi
\IfFileExists{bookmark.sty}{\usepackage{bookmark}}{\usepackage{hyperref}}
\IfFileExists{xurl.sty}{\usepackage{xurl}}{} % add URL line breaks if available
\urlstyle{same}
\hypersetup{
  pdftitle={Modèles de durée : TD et Examens},
  hidelinks,
  pdfcreator={LaTeX via pandoc}}

\title{Modèles de durée : TD et Examens}
\author{}
\date{\vspace{-2.5em}2024-11-10}

\begin{document}
\maketitle

{
\setcounter{tocdepth}{3}
\tableofcontents
}
\hypertarget{les-muxe9thodes-semi-paramuxe9triques}{%
\section{Les méthodes
semi-paramétriques}\label{les-muxe9thodes-semi-paramuxe9triques}}

\hypertarget{le-moduxe8le-de-cox}{%
\subsection{Le modèle de Cox}\label{le-moduxe8le-de-cox}}

\hypertarget{lecture-des-donnuxe9es-traitement-de-la-base}{%
\subsubsection{Lecture des données traitement de la base
:}\label{lecture-des-donnuxe9es-traitement-de-la-base}}

\begin{Shaded}
\begin{Highlighting}[]
\FunctionTok{library}\NormalTok{(tidyverse)}
\NormalTok{Re }\OtherTok{=} \FunctionTok{read.table}\NormalTok{(}\StringTok{"rossi.txt"}\NormalTok{, }\AttributeTok{header =} \ConstantTok{TRUE}\NormalTok{)}
\FunctionTok{glimpse}\NormalTok{(Re)}

\CommentTok{\# Suppression de la variable race : }
\NormalTok{Re1 }\OtherTok{=}\NormalTok{ Re[, }\SpecialCharTok{{-}}\DecValTok{5}\NormalTok{]}
\end{Highlighting}
\end{Shaded}

\hypertarget{etude-la-duruxe9e-de-survie-selon-la-valeur-dune-variable.-test-de-log-rank}{%
\subsubsection{Etude la durée de survie selon la valeur d'une variable.
(test de
log-Rank)}\label{etude-la-duruxe9e-de-survie-selon-la-valeur-dune-variable.-test-de-log-rank}}

On regarde si les fonctions de survies des individus discriminés selon
les modalités d'une variable, sont significativement similaires.

On effectue pour ça le test du log-rank à l'aide de la fonction Surv du
package survival.

\[
\left\{
\begin{array}{l}
H0 : \text{les fonctions de survie sont les mêmes,  p-value $\geq$ 0.05} \\
H1 : \text{les fonctions de survie sont différentes} \\
\end{array}
\right.
\]

\begin{Shaded}
\begin{Highlighting}[]
\FunctionTok{library}\NormalTok{(survival)}
\CommentTok{\# Test sur la variable financement : }
\FunctionTok{survdiff}\NormalTok{(}\FunctionTok{Surv}\NormalTok{(week, arrest) }\SpecialCharTok{\textasciitilde{}}\NormalTok{ fin, }\AttributeTok{data =}\NormalTok{ Re1)}
\end{Highlighting}
\end{Shaded}

\begin{verbatim}
Call:
survdiff(formula = Surv(week, arrest) ~ fin, data = Re1)

        N Observed Expected (O-E)^2/E (O-E)^2/V
fin=0 216       66     55.6      1.96      3.84
fin=1 216       48     58.4      1.86      3.84

 Chisq= 3.8  on 1 degrees of freedom, p= 0.05 
\end{verbatim}

\begin{Shaded}
\begin{Highlighting}[]
\CommentTok{\# Surv créer un objet avec week le temps de survie et arrest l\textquotesingle{}indicateur }
\CommentTok{\# d’évènement. Fin est la variable servant à comparer les courbes. }
\end{Highlighting}
\end{Shaded}

\hypertarget{moduxe9lisation-de-kaplan-meier}{%
\subsubsection{Modélisation de Kaplan Meier
:}\label{moduxe9lisation-de-kaplan-meier}}

\begin{Shaded}
\begin{Highlighting}[]
\CommentTok{\# Modélisation de kaplan meier, distinction sur la variable financement }
\NormalTok{s }\OtherTok{=} \FunctionTok{survfit}\NormalTok{(}\FunctionTok{Surv}\NormalTok{(week, arrest) }\SpecialCharTok{\textasciitilde{}}\NormalTok{ fin, }\AttributeTok{data =}\NormalTok{ Re1)}
\FunctionTok{library}\NormalTok{(ggfortify)}
\FunctionTok{library}\NormalTok{(ggplot2)}
\FunctionTok{autoplot}\NormalTok{(s) }
\end{Highlighting}
\end{Shaded}

\begin{center}\includegraphics[width=0.95\linewidth]{1-MEMO_MODELE_DUREE_files/figure-latex/unnamed-chunk-3-1} \end{center}

\hypertarget{ajustement-dun-moduxe8le-de-cox}{%
\subsubsection{Ajustement d'un modèle de Cox
:}\label{ajustement-dun-moduxe8le-de-cox}}

\begin{Shaded}
\begin{Highlighting}[]
\NormalTok{cox1 }\OtherTok{=} \FunctionTok{coxph}\NormalTok{(}\AttributeTok{formula =} \FunctionTok{Surv}\NormalTok{(week, arrest) }\SpecialCharTok{\textasciitilde{}}\NormalTok{ fin }\SpecialCharTok{+}\NormalTok{ age }\SpecialCharTok{+}\NormalTok{ wexp }\SpecialCharTok{+}\NormalTok{ mar }\SpecialCharTok{+} 
\NormalTok{               paro }\SpecialCharTok{+}\NormalTok{ prio, }\AttributeTok{data =}\NormalTok{ Re1)}
\FunctionTok{summary}\NormalTok{(cox1)}
\end{Highlighting}
\end{Shaded}

\begin{verbatim}
Call:
coxph(formula = Surv(week, arrest) ~ fin + age + wexp + mar + 
    paro + prio, data = Re1)

  n= 432, number of events= 114 

         coef exp(coef) se(coef)      z Pr(>|z|)   
fin  -0.36554   0.69382  0.19090 -1.915  0.05552 . 
age  -0.05633   0.94523  0.02189 -2.573  0.01007 * 
wexp -0.15699   0.85471  0.21208 -0.740  0.45916   
mar  -0.47130   0.62419  0.38027 -1.239  0.21520   
paro -0.07792   0.92504  0.19530 -0.399  0.68991   
prio  0.08966   1.09380  0.02871  3.123  0.00179 **
---
Signif. codes:  0 '***' 0.001 '**' 0.01 '*' 0.05 '.' 0.1 ' ' 1

     exp(coef) exp(-coef) lower .95 upper .95
fin     0.6938     1.4413    0.4773    1.0087
age     0.9452     1.0579    0.9055    0.9867
wexp    0.8547     1.1700    0.5640    1.2952
mar     0.6242     1.6021    0.2962    1.3152
paro    0.9250     1.0810    0.6308    1.3564
prio    1.0938     0.9142    1.0340    1.1571

Concordance= 0.639  (se = 0.027 )
Likelihood ratio test= 32.14  on 6 df,   p=2e-05
Wald test            = 30.79  on 6 df,   p=3e-05
Score (logrank) test = 32.28  on 6 df,   p=1e-05
\end{verbatim}

Explication du test :

\[
\left\{
\begin{array}{l}
H0 : \text{ $\beta_j$ = 0,  Pr(>|z|), prob(|U|> z), où U ~ N(0,1)} \\
H1 : \text{$beta_j \neq 0$, p-value $\leq$ 0.05} \\
\end{array}
\right.
\] Le se(coef) correspond au sqrt(var(beta)). On en déduit que les
variables significatives sont l'âge et le prio.

\hypertarget{graphique-de-la-fonction-de-survie}{%
\subsubsection{Graphique de la fonction de survie
:}\label{graphique-de-la-fonction-de-survie}}

Dans le cadre des fonction de Kaplan Meier, Aalen par défaut les
covariables sont fixées à la valeur moyenne.

\begin{Shaded}
\begin{Highlighting}[]
\NormalTok{kpmr }\OtherTok{=} \FunctionTok{survfit}\NormalTok{(cox1) }\CommentTok{\# Fonction de survie de Kaplan{-}Meier pour le modèle de cox}

\FunctionTok{summary}\NormalTok{(kpmr)}
\end{Highlighting}
\end{Shaded}

\begin{verbatim}
Call: survfit(formula = cox1)

 time n.risk n.event survival std.err lower 95% CI upper 95% CI
    1    432       1    0.997 0.00292        0.991        1.000
    2    431       1    0.994 0.00419        0.986        1.000
    3    430       1    0.991 0.00520        0.981        1.000
    4    429       1    0.989 0.00609        0.977        1.000
    5    428       1    0.986 0.00690        0.972        0.999
    6    427       1    0.983 0.00766        0.968        0.998
    7    426       1    0.980 0.00838        0.964        0.997
    8    425       5    0.966 0.01165        0.943        0.989
    9    420       2    0.960 0.01285        0.935        0.985
   10    418       1    0.957 0.01343        0.931        0.984
   11    417       2    0.951 0.01459        0.923        0.980
   12    415       2    0.945 0.01573        0.915        0.977
   13    413       1    0.943 0.01629        0.911        0.975
   14    412       3    0.934 0.01794        0.899        0.970
   15    409       2    0.928 0.01903        0.891        0.966
   16    407       2    0.922 0.02009        0.884        0.962
   17    405       3    0.913 0.02167        0.872        0.957
   18    402       3    0.905 0.02322        0.860        0.951
   19    399       2    0.899 0.02424        0.853        0.948
   20    397       5    0.884 0.02674        0.833        0.938
   21    392       2    0.878 0.02772        0.826        0.934
   22    390       1    0.875 0.02820        0.822        0.933
   23    389       1    0.873 0.02868        0.818        0.931
   24    388       4    0.861 0.03057        0.803        0.923
   25    384       3    0.852 0.03196        0.792        0.917
   26    381       3    0.843 0.03332        0.781        0.911
   27    378       2    0.838 0.03422        0.773        0.907
   28    376       2    0.832 0.03512        0.766        0.904
   30    374       2    0.826 0.03601        0.758        0.900
   31    372       1    0.823 0.03645        0.755        0.898
   32    371       2    0.817 0.03732        0.747        0.894
   33    369       2    0.811 0.03819        0.740        0.890
   34    367       2    0.805 0.03906        0.732        0.886
   35    365       4    0.794 0.04077        0.718        0.878
   36    361       3    0.785 0.04202        0.707        0.872
   37    358       4    0.773 0.04365        0.692        0.864
   38    354       1    0.770 0.04405        0.689        0.862
   39    353       2    0.764 0.04485        0.681        0.858
   40    351       4    0.753 0.04641        0.667        0.849
   42    347       2    0.747 0.04717        0.660        0.845
   43    345       4    0.735 0.04867        0.646        0.837
   44    341       2    0.729 0.04941        0.639        0.833
   45    339       2    0.724 0.05014        0.632        0.829
   46    337       4    0.712 0.05157        0.618        0.820
   47    333       1    0.709 0.05191        0.614        0.818
   48    332       2    0.703 0.05261        0.607        0.814
   49    330       5    0.689 0.05430        0.590        0.804
   50    325       3    0.680 0.05527        0.580        0.797
   52    322       4    0.668 0.05653        0.566        0.789
\end{verbatim}

\begin{Shaded}
\begin{Highlighting}[]
\FunctionTok{plot}\NormalTok{(}
\NormalTok{  kpmr,}
  \AttributeTok{ylim =} \FunctionTok{c}\NormalTok{(}\FloatTok{0.5}\NormalTok{, }\DecValTok{1}\NormalTok{),}
  \AttributeTok{lty =} \DecValTok{5}\NormalTok{,}
  \AttributeTok{xlab =} \StringTok{\textquotesingle{}Semaine\textquotesingle{}}\NormalTok{,}
  \AttributeTok{ylab =} \StringTok{\textquotesingle{}Proportion de non recidive\textquotesingle{}}\NormalTok{,}
  \AttributeTok{main =} \StringTok{\textquotesingle{}Fonction de survie estimation de Kaplan{-}Meier\textquotesingle{}}\NormalTok{,}
  \AttributeTok{col =}\NormalTok{ palette\_couleur[}\DecValTok{1}\SpecialCharTok{:}\DecValTok{3}\NormalTok{], }
  \AttributeTok{lwd =} \DecValTok{2}\NormalTok{)}

\FunctionTok{legend}\NormalTok{(}
  \StringTok{"topright"}\NormalTok{,  }\CommentTok{\# Position de la légende}
  \AttributeTok{lty =} \DecValTok{1}\NormalTok{,}
  \AttributeTok{cex =} \DecValTok{1}\NormalTok{,}
  \AttributeTok{legend =} \FunctionTok{c}\NormalTok{(}\StringTok{"KP"}\NormalTok{, }\StringTok{"lower"}\NormalTok{, }\StringTok{"upper"}\NormalTok{),}
  \AttributeTok{col =}\NormalTok{ palette\_couleur[}\DecValTok{1}\SpecialCharTok{:}\DecValTok{3}\NormalTok{])}
\end{Highlighting}
\end{Shaded}

\begin{center}\includegraphics[width=0.95\linewidth]{1-MEMO_MODELE_DUREE_files/figure-latex/unnamed-chunk-5-1} \end{center}

\hypertarget{fonction-de-hasard-cumuluxe9e-avec-lestimateur-de-breslow}{%
\subsubsection{Fonction de hasard cumulée avec l'estimateur de Breslow
:}\label{fonction-de-hasard-cumuluxe9e-avec-lestimateur-de-breslow}}

Interprétation Intuitive:

Taux Instantané: La fonction de hasard représente le taux instantané de
survenue de l'événement à un moment donné.\\
Par exemple, si h(t)=0.05h(t)=0.05 à t=10t=10 semaines, cela signifie
que le taux de survenue de l'événement à 10 semaines est de 5\% par
unité de temps.\\
Conditionnelle à la Survie: La fonction de hasard est conditionnelle à
la survie jusqu'à ce moment. Elle ne prend en compte que les individus
qui n'ont pas encore subi l'événement.

\begin{Shaded}
\begin{Highlighting}[]
\FunctionTok{plot}\NormalTok{(}
  \FunctionTok{basehaz}\NormalTok{(cox1),}
  \AttributeTok{main =} \StringTok{\textquotesingle{}Fonction de hasard de baseline\textquotesingle{}}\NormalTok{,}
  \AttributeTok{xlab =} \StringTok{\textquotesingle{}Valeur de la fonction de hasard\textquotesingle{}}\NormalTok{,}
  \AttributeTok{ylab =} \StringTok{"Temsp écoulé"}\NormalTok{,}
  \AttributeTok{type =} \StringTok{\textquotesingle{}l\textquotesingle{}}\NormalTok{,}
  \AttributeTok{col =}\NormalTok{ palette\_couleur[}\DecValTok{1}\NormalTok{],}
  \AttributeTok{lwd =} \DecValTok{3}
\NormalTok{)}
\end{Highlighting}
\end{Shaded}

\begin{center}\includegraphics[width=0.95\linewidth]{1-MEMO_MODELE_DUREE_files/figure-latex/unnamed-chunk-6-1} \end{center}

\hypertarget{fonction-survie-pour-lindividu-ayant-les-caractuxe9ristiques-du-premier-individu}{%
\subsubsection{Fonction survie pour l'individu ayant les
caractéristiques du premier individu
:}\label{fonction-survie-pour-lindividu-ayant-les-caractuxe9ristiques-du-premier-individu}}

\begin{Shaded}
\begin{Highlighting}[]
\CommentTok{\# plot(survfit(cox1, newdata = Re1)) \# fonction de survie pour tous les individus}
\CommentTok{\# title("Fonction de survie pour tous les individus")}

\FunctionTok{plot}\NormalTok{(}\FunctionTok{survfit}\NormalTok{(cox1, }\AttributeTok{newdata =}\NormalTok{ Re1[}\DecValTok{1}\NormalTok{, ]),}
     \AttributeTok{main =} \StringTok{"Fonction de survie pour un individu donné"}\NormalTok{,}
     \AttributeTok{col =}\NormalTok{ palette\_couleur[}\DecValTok{1}\SpecialCharTok{:}\DecValTok{3}\NormalTok{], }
     \AttributeTok{ylim =} \FunctionTok{c}\NormalTok{(}\FloatTok{0.5}\NormalTok{,}\DecValTok{1}\NormalTok{), }
     \AttributeTok{lwd =} \DecValTok{2}\NormalTok{, }
     \AttributeTok{lty =} \DecValTok{1}\NormalTok{)}
 
\FunctionTok{legend}\NormalTok{(}\StringTok{"bottomleft"}\NormalTok{,}
       \AttributeTok{lty =} \DecValTok{1}\NormalTok{,}
       \AttributeTok{cex =} \DecValTok{1}\NormalTok{,}
       \AttributeTok{legend =} \FunctionTok{c}\NormalTok{(}\StringTok{"KP"}\NormalTok{, }\StringTok{"lower"}\NormalTok{, }\StringTok{"upper"}\NormalTok{),}
       \AttributeTok{col =}\NormalTok{ palette\_couleur[}\DecValTok{1}\SpecialCharTok{:}\DecValTok{3}\NormalTok{])}
\end{Highlighting}
\end{Shaded}

\begin{center}\includegraphics[width=0.95\linewidth]{1-MEMO_MODELE_DUREE_files/figure-latex/unnamed-chunk-7-1} \end{center}

\hypertarget{etude-de-leffet-dune-covariable-les-autres-uxe9tant-fixuxe9es}{%
\subsubsection{Etude de l'effet d'une covariable (les autres étant
fixées)
:}\label{etude-de-leffet-dune-covariable-les-autres-uxe9tant-fixuxe9es}}

Exemple : effet de la var ``financement'' (0 ou 1) On fixe les autres à
leur valeur moyenne.

\begin{Shaded}
\begin{Highlighting}[]
\NormalTok{ReFin }\OtherTok{=} \FunctionTok{data.frame}\NormalTok{(}
  \AttributeTok{fin =} \FunctionTok{c}\NormalTok{(}\DecValTok{0}\NormalTok{, }\DecValTok{1}\NormalTok{),}
  \AttributeTok{age =} \FunctionTok{rep}\NormalTok{(}\FunctionTok{mean}\NormalTok{(Re1}\SpecialCharTok{$}\NormalTok{age), }\DecValTok{2}\NormalTok{),}
  \AttributeTok{wexp =} \FunctionTok{rep}\NormalTok{(}\FunctionTok{mean}\NormalTok{(Re1}\SpecialCharTok{$}\NormalTok{wexp), }\DecValTok{2}\NormalTok{),}
  \AttributeTok{mar =} \FunctionTok{rep}\NormalTok{(}\FunctionTok{mean}\NormalTok{(Re1}\SpecialCharTok{$}\NormalTok{mar), }\DecValTok{2}\NormalTok{),}
  \AttributeTok{paro =} \FunctionTok{rep}\NormalTok{(}\FunctionTok{mean}\NormalTok{(Re1}\SpecialCharTok{$}\NormalTok{paro), }\DecValTok{2}\NormalTok{),}
  \AttributeTok{prio =} \FunctionTok{rep}\NormalTok{(}\FunctionTok{mean}\NormalTok{(Re1}\SpecialCharTok{$}\NormalTok{prio), }\DecValTok{2}\NormalTok{)}
\NormalTok{)}

\FunctionTok{plot}\NormalTok{(}
  \FunctionTok{survfit}\NormalTok{(cox1, }\AttributeTok{newdata =}\NormalTok{ ReFin),}
  \AttributeTok{lty =} \FunctionTok{c}\NormalTok{(}\DecValTok{1}\NormalTok{, }\DecValTok{2}\NormalTok{),}
  \AttributeTok{ylim =} \FunctionTok{c}\NormalTok{(.}\DecValTok{6}\NormalTok{, }\DecValTok{1}\NormalTok{),}
  \AttributeTok{col =}\NormalTok{ palette\_couleur[}\DecValTok{4}\SpecialCharTok{:}\DecValTok{5}\NormalTok{],}
  \AttributeTok{lwd =} \DecValTok{2}\NormalTok{, }
  \AttributeTok{main =} \StringTok{"Fonction de survie selon la modalité de financement"}\NormalTok{, }
  \AttributeTok{ylab =} \StringTok{"Probabilité de survie estimée"}\NormalTok{, }
  \AttributeTok{xlab =} \StringTok{"Période de temps écoulée"}
\NormalTok{)}
\FunctionTok{legend}\NormalTok{(}
  \DecValTok{1}\NormalTok{,}
  \FloatTok{0.8}\NormalTok{,}
  \AttributeTok{legend =} \FunctionTok{c}\NormalTok{(}\StringTok{"fin=0"}\NormalTok{, }\StringTok{"fin=1"}\NormalTok{),}
  \AttributeTok{lty =} \FunctionTok{c}\NormalTok{(}\DecValTok{1}\NormalTok{, }\DecValTok{2}\NormalTok{),}
  \AttributeTok{col =}\NormalTok{  palette\_couleur[}\DecValTok{4}\SpecialCharTok{:}\DecValTok{5}\NormalTok{]}
\NormalTok{)}
\end{Highlighting}
\end{Shaded}

\begin{center}\includegraphics[width=0.95\linewidth]{1-MEMO_MODELE_DUREE_files/figure-latex/unnamed-chunk-8-1} \end{center}

\hypertarget{suxe9lection-de-variable-une-uxe0-une}{%
\subsubsection{Sélection de variable une à une
:}\label{suxe9lection-de-variable-une-uxe0-une}}

Remarque : on peut faire de la sélection de variables en enlevant de
façon itérative celles expliquant le moins (p-value la plus forte)
exemple :

\begin{Shaded}
\begin{Highlighting}[]
\NormalTok{cox2}\OtherTok{=} \FunctionTok{coxph}\NormalTok{(}\AttributeTok{formula=}\FunctionTok{Surv}\NormalTok{(week,arrest)}\SpecialCharTok{\textasciitilde{}}\NormalTok{fin}\SpecialCharTok{+}\NormalTok{age}\SpecialCharTok{+}\NormalTok{wexp}\SpecialCharTok{+}\NormalTok{mar}\SpecialCharTok{+}\NormalTok{prio,}\AttributeTok{data=}\NormalTok{Re1)}
\FunctionTok{summary}\NormalTok{(cox2)}
\end{Highlighting}
\end{Shaded}

\begin{verbatim}
Call:
coxph(formula = Surv(week, arrest) ~ fin + age + wexp + mar + 
    prio, data = Re1)

  n= 432, number of events= 114 

         coef exp(coef) se(coef)      z Pr(>|z|)   
fin  -0.36094   0.69702  0.19052 -1.894   0.0582 . 
age  -0.05536   0.94614  0.02172 -2.549   0.0108 * 
wexp -0.16039   0.85181  0.21201 -0.757   0.4493   
mar  -0.47935   0.61919  0.37989 -1.262   0.2070   
prio  0.09134   1.09564  0.02840  3.216   0.0013 **
---
Signif. codes:  0 '***' 0.001 '**' 0.01 '*' 0.05 '.' 0.1 ' ' 1

     exp(coef) exp(-coef) lower .95 upper .95
fin     0.6970     1.4347    0.4798    1.0126
age     0.9461     1.0569    0.9067    0.9873
wexp    0.8518     1.1740    0.5622    1.2906
mar     0.6192     1.6150    0.2941    1.3037
prio    1.0956     0.9127    1.0363    1.1583

Concordance= 0.641  (se = 0.027 )
Likelihood ratio test= 31.98  on 5 df,   p=6e-06
Wald test            = 30.73  on 5 df,   p=1e-05
Score (logrank) test = 32.2  on 5 df,   p=5e-06
\end{verbatim}

Test hypothèse de Hasard Proportionnel : (proportionnalité des risques)
\[
\left\{
\begin{array}{l}
H0 : \text{les résidus sont indépendants du temps} \\
H1 : \text{les résidus dépendent du temps} \\
\end{array}
\right.
\]

Explication : Si H0 est rejetée, alors les résidus dépendent du temps

\hypertarget{test-de-hasard-proportionnel-les-ruxe9sidus-de-schoenfeld}{%
\subsubsection{Test de hasard proportionnel, les résidus de
Schoenfeld}\label{test-de-hasard-proportionnel-les-ruxe9sidus-de-schoenfeld}}

\begin{Shaded}
\begin{Highlighting}[]
\NormalTok{res }\OtherTok{=} \FunctionTok{cox.zph}\NormalTok{(cox1)}
\NormalTok{res}
\end{Highlighting}
\end{Shaded}

\begin{verbatim}
         chisq df     p
fin     0.0621  1 0.803
age     5.9161  1 0.015
wexp    4.2983  1 0.038
mar     1.0207  1 0.312
paro    0.0140  1 0.906
prio    0.5254  1 0.469
GLOBAL 16.4474  6 0.012
\end{verbatim}

\begin{Shaded}
\begin{Highlighting}[]
\CommentTok{\# Représentation graphique}
\FunctionTok{par}\NormalTok{(}\AttributeTok{mfrow =} \FunctionTok{c}\NormalTok{(}\DecValTok{2}\NormalTok{, }\DecValTok{4}\NormalTok{))}
\FunctionTok{plot}\NormalTok{(res)}
\end{Highlighting}
\end{Shaded}

\begin{center}\includegraphics[width=0.95\linewidth]{1-MEMO_MODELE_DUREE_files/figure-latex/unnamed-chunk-10-1} \end{center}

\hypertarget{les-muxe9thodes-non-paramuxe9triques}{%
\section{Les méthodes
non-paramétriques}\label{les-muxe9thodes-non-paramuxe9triques}}

\hypertarget{la-muxe9thode-de-kaplan-meier}{%
\subsection{La méthode de Kaplan meier
:}\label{la-muxe9thode-de-kaplan-meier}}

\hypertarget{guxe9nuxe9ration-de-la-base-et-importation-des-donnuxe9es}{%
\subsubsection{Génération de la base et importation des
données}\label{guxe9nuxe9ration-de-la-base-et-importation-des-donnuxe9es}}

On créer une base de données avec des observations censurées.

\begin{Shaded}
\begin{Highlighting}[]
\FunctionTok{library}\NormalTok{(survival)}
\NormalTok{tempsGMP }\OtherTok{=} \FunctionTok{c}\NormalTok{(}\FunctionTok{rep}\NormalTok{(}\DecValTok{6}\NormalTok{, }\DecValTok{4}\NormalTok{), }\DecValTok{7}\NormalTok{, }\DecValTok{9}\NormalTok{, }\DecValTok{10}\NormalTok{, }\DecValTok{10}\NormalTok{, }\DecValTok{11}\NormalTok{, }\DecValTok{13}\NormalTok{, }\DecValTok{16}\NormalTok{, }\DecValTok{17}\NormalTok{, }\DecValTok{19}\NormalTok{, }\DecValTok{20}\NormalTok{, }\DecValTok{22}\NormalTok{, }\DecValTok{23}\NormalTok{, }\DecValTok{25}\NormalTok{, }\DecValTok{32}\NormalTok{, }
             \DecValTok{32}\NormalTok{, }\DecValTok{34}\NormalTok{, }\DecValTok{35}\NormalTok{) }\CommentTok{\# liste des observations}
\NormalTok{finGMP }\OtherTok{=} \FunctionTok{c}\NormalTok{(}\DecValTok{1}\NormalTok{, }\DecValTok{1}\NormalTok{, }\DecValTok{1}\NormalTok{, }\DecValTok{0}\NormalTok{, }\DecValTok{1}\NormalTok{, }\DecValTok{0}\NormalTok{, }\DecValTok{1}\NormalTok{, }\DecValTok{0}\NormalTok{, }\DecValTok{0}\NormalTok{, }\DecValTok{1}\NormalTok{, }\DecValTok{1}\NormalTok{, }\DecValTok{0}\NormalTok{, }\DecValTok{0}\NormalTok{, }\DecValTok{0}\NormalTok{, }\DecValTok{1}\NormalTok{, }\DecValTok{1}\NormalTok{, }\FunctionTok{rep}\NormalTok{(}\DecValTok{0}\NormalTok{, }\DecValTok{5}\NormalTok{)) }
\CommentTok{\# indication des obs censurées}
\NormalTok{donnF }\OtherTok{=} \FunctionTok{Surv}\NormalTok{(tempsGMP, finGMP) }
\FunctionTok{head}\NormalTok{(donnF)}
\end{Highlighting}
\end{Shaded}

\begin{verbatim}
[1] 6  6  6  6+ 7  9+
\end{verbatim}

\hypertarget{ajustement-dun-moduxe8le-de-survie-avec-la-muxe9thode-de-kaplan-meier}{%
\subsubsection{Ajustement d'un modèle de survie avec la méthode de
Kaplan Meier
:}\label{ajustement-dun-moduxe8le-de-survie-avec-la-muxe9thode-de-kaplan-meier}}

\begin{Shaded}
\begin{Highlighting}[]
\NormalTok{survKM }\OtherTok{=} \FunctionTok{survfit}\NormalTok{(donnF }\SpecialCharTok{\textasciitilde{}} \DecValTok{1}\NormalTok{,}
                 \AttributeTok{data =}\NormalTok{ donnF,}
                 \AttributeTok{type =} \StringTok{"kaplan{-}meier"}\NormalTok{,}
                 \AttributeTok{error =} \StringTok{"greenwood"}\NormalTok{)}

\CommentTok{\# Graphique de la fonction de survie moyenne : }

\FunctionTok{plot}\NormalTok{(survKM, }\AttributeTok{mark.time =} \ConstantTok{TRUE}\NormalTok{, }\AttributeTok{col =}\NormalTok{ palette\_couleur[}\DecValTok{1}\SpecialCharTok{:}\DecValTok{3}\NormalTok{])}
\FunctionTok{title}\NormalTok{(}\StringTok{"Modèle de survie de Kaplan{-}Meier"}\NormalTok{)}
\FunctionTok{legend}\NormalTok{(}\StringTok{"bottomleft"}\NormalTok{,  }
       \AttributeTok{lty =} \DecValTok{1}\NormalTok{,}
       \AttributeTok{cex =} \DecValTok{1}\NormalTok{,}
       \AttributeTok{legend =} \FunctionTok{c}\NormalTok{(}\StringTok{"KP"}\NormalTok{, }\StringTok{"lower"}\NormalTok{, }\StringTok{"upper"}\NormalTok{),}
       \AttributeTok{col =}\NormalTok{ palette\_couleur[}\DecValTok{1}\SpecialCharTok{:}\DecValTok{3}\NormalTok{])}
\end{Highlighting}
\end{Shaded}

\begin{center}\includegraphics[width=0.95\linewidth]{1-MEMO_MODELE_DUREE_files/figure-latex/unnamed-chunk-12-1} \end{center}

\begin{Shaded}
\begin{Highlighting}[]
\CommentTok{\# Intervalle de confiance et valeur modélisée pour l\textquotesingle{}individu 10 : }
\NormalTok{IC\_KM }\OtherTok{=} \FunctionTok{round}\NormalTok{(}\FunctionTok{c}\NormalTok{(survKM}\SpecialCharTok{$}\NormalTok{lower[}\DecValTok{10}\NormalTok{], survKM}\SpecialCharTok{$}\NormalTok{surv[}\DecValTok{10}\NormalTok{], survKM}\SpecialCharTok{$}\NormalTok{upper[}\DecValTok{10}\NormalTok{]),}\DecValTok{4}\NormalTok{)}
\end{Highlighting}
\end{Shaded}

\hypertarget{le-moduxe8le-de-fleming-harrington}{%
\subsection{Le modèle de Fleming-Harrington
:}\label{le-moduxe8le-de-fleming-harrington}}

\hypertarget{moduxe8le-de-fleming-harrington-intervalle-muxe9thode-tsiatis}{%
\subsubsection{Modèle de Fleming-Harrington, intervalle méthode Tsiatis
:}\label{moduxe8le-de-fleming-harrington-intervalle-muxe9thode-tsiatis}}

\begin{Shaded}
\begin{Highlighting}[]
\CommentTok{\# Par defaut, intervalle de confiance : conf.type=\textquotesingle{}log\textquotesingle{} :}
\NormalTok{survFH }\OtherTok{=} \FunctionTok{survfit}\NormalTok{(donnF }\SpecialCharTok{\textasciitilde{}} \DecValTok{1}\NormalTok{,}
                 \AttributeTok{data =}\NormalTok{ donnF,}
                 \AttributeTok{type =} \StringTok{"fleming{-}harrington"}\NormalTok{,}
                 \AttributeTok{error =} \StringTok{"tsiatis"}\NormalTok{)}

\CommentTok{\# Intervalle de confiance et valeur modélisée pour l\textquotesingle{}individu 10 : }
\NormalTok{IC\_FH }\OtherTok{=} \FunctionTok{round}\NormalTok{(}\FunctionTok{c}\NormalTok{(survFH}\SpecialCharTok{$}\NormalTok{lower[}\DecValTok{10}\NormalTok{], survFH}\SpecialCharTok{$}\NormalTok{surv[}\DecValTok{10}\NormalTok{], survFH}\SpecialCharTok{$}\NormalTok{upper[}\DecValTok{10}\NormalTok{]),}\DecValTok{4}\NormalTok{)}
\end{Highlighting}
\end{Shaded}

\hypertarget{moduxe8le-de-fleming-harrington-intervalle-muxe9thode-delta}{%
\subsubsection{Modèle de Fleming-Harrington, intervalle méthode delta
:}\label{moduxe8le-de-fleming-harrington-intervalle-muxe9thode-delta}}

\begin{Shaded}
\begin{Highlighting}[]
\NormalTok{survFHdelta }\OtherTok{=} \FunctionTok{survfit}\NormalTok{(}
\NormalTok{  donnF }\SpecialCharTok{\textasciitilde{}} \DecValTok{1}\NormalTok{,}
  \AttributeTok{data =}\NormalTok{ donnF,}
  \AttributeTok{type =} \StringTok{"fleming{-}harrington"}\NormalTok{,}
  \AttributeTok{error =} \StringTok{"tsiatis"}\NormalTok{,}
  \AttributeTok{conf.type =} \StringTok{"plain"}\NormalTok{)}

\NormalTok{IC\_FHdelta }\OtherTok{=} \FunctionTok{round}\NormalTok{(}\FunctionTok{c}\NormalTok{(survFHdelta}\SpecialCharTok{$}\NormalTok{lower[}\DecValTok{10}\NormalTok{], survFHdelta}\SpecialCharTok{$}\NormalTok{surv[}\DecValTok{10}\NormalTok{], survFHdelta}\SpecialCharTok{$}\NormalTok{upper[}\DecValTok{10}\NormalTok{]),}\DecValTok{4}\NormalTok{)}
\end{Highlighting}
\end{Shaded}

\hypertarget{comparaison-des-ruxe9sultats-sur-lestimation-du-10e-individu-de-la-base}{%
\subsubsection{Comparaison des résultats sur l'estimation du 10e
individu de la base
:}\label{comparaison-des-ruxe9sultats-sur-lestimation-du-10e-individu-de-la-base}}

\begin{Shaded}
\begin{Highlighting}[]
\CommentTok{\#Comparaison des modèles pour le 10e individu de la base }
\NormalTok{dt }\OtherTok{=} \FunctionTok{data.frame}\NormalTok{(}\AttributeTok{KM =}\NormalTok{ IC\_KM, }\AttributeTok{FH =}\NormalTok{ IC\_FH, }\AttributeTok{FHdelta =}\NormalTok{ IC\_FHdelta)}
\FunctionTok{rownames}\NormalTok{(dt) }\OtherTok{=} \FunctionTok{c}\NormalTok{(}\StringTok{"lower"}\NormalTok{,}\StringTok{"pred"}\NormalTok{,}\StringTok{"upper"}\NormalTok{)}
\NormalTok{dt}
\end{Highlighting}
\end{Shaded}

\begin{verbatim}
          KM     FH FHdelta
lower 0.4394 0.4577  0.4246
pred  0.6275 0.6424  0.6424
upper 0.8960 0.9016  0.8601
\end{verbatim}

\hypertarget{repruxe9sentation-graphiques-des-trois-moduxe8les}{%
\subsubsection{Représentation graphiques des trois modèles
:}\label{repruxe9sentation-graphiques-des-trois-moduxe8les}}

\begin{Shaded}
\begin{Highlighting}[]
\CommentTok{\# Graphiques des trois modèles : }
\FunctionTok{plot}\NormalTok{(}
\NormalTok{  survKM,}
  \AttributeTok{mark.time =} \ConstantTok{TRUE}\NormalTok{,}
  \AttributeTok{col =}\NormalTok{ palette\_couleur[}\DecValTok{1}\NormalTok{],}
  \AttributeTok{lwd =} \DecValTok{2}\NormalTok{,}
  \AttributeTok{xlab =} \StringTok{"Durée de survie"}\NormalTok{,}
  \AttributeTok{ylab =} \StringTok{"Probabilité de survie"}
\NormalTok{)}
\FunctionTok{lines}\NormalTok{(survFH,}
      \AttributeTok{mark.time =} \ConstantTok{TRUE}\NormalTok{,}
      \AttributeTok{col =}\NormalTok{ palette\_couleur[}\DecValTok{2}\NormalTok{],}
      \AttributeTok{lwd =} \DecValTok{2}\NormalTok{)}
\FunctionTok{lines}\NormalTok{(survFHdelta,}
      \AttributeTok{mark.time =} \ConstantTok{TRUE}\NormalTok{,}
      \AttributeTok{col =}\NormalTok{ palette\_couleur[}\DecValTok{3}\NormalTok{],}
      \AttributeTok{lwd =} \DecValTok{2}\NormalTok{)}
\FunctionTok{title}\NormalTok{(}\StringTok{"Comparaison des modèles de survie"}\NormalTok{)}
\FunctionTok{legend}\NormalTok{(}
  \DecValTok{1}\NormalTok{,}
  \FloatTok{0.4}\NormalTok{,}
  \AttributeTok{lty =} \DecValTok{1}\NormalTok{,}
  \AttributeTok{cex =} \DecValTok{1}\NormalTok{,}
  \AttributeTok{legend =} \FunctionTok{c}\NormalTok{(}\StringTok{"KM"}\NormalTok{, }\StringTok{"FH"}\NormalTok{, }\StringTok{"FHdelta"}\NormalTok{),}
  \AttributeTok{col =}\NormalTok{ palette\_couleur[}\DecValTok{1}\SpecialCharTok{:}\DecValTok{3}\NormalTok{]}
\NormalTok{)}
\end{Highlighting}
\end{Shaded}

\begin{center}\includegraphics[width=0.95\linewidth]{1-MEMO_MODELE_DUREE_files/figure-latex/unnamed-chunk-16-1} \end{center}

\hypertarget{estimation-par-des-lois-usuelles}{%
\subsection{Estimation par des lois usuelles
:}\label{estimation-par-des-lois-usuelles}}

\hypertarget{estimation-de-la-loi-de-x-par-une-loi-de-weibull}{%
\subsubsection{Estimation de la loi de X par une loi de Weibull
:}\label{estimation-de-la-loi-de-x-par-une-loi-de-weibull}}

\begin{Shaded}
\begin{Highlighting}[]
\NormalTok{survweib }\OtherTok{=} \FunctionTok{survreg}\NormalTok{(donnF }\SpecialCharTok{\textasciitilde{}} \DecValTok{1}\NormalTok{, }\AttributeTok{dist =} \StringTok{"weibull"}\NormalTok{)}
\NormalTok{survweib}
\end{Highlighting}
\end{Shaded}

\begin{verbatim}
Call:
survreg(formula = donnF ~ 1, dist = "weibull")

Coefficients:
(Intercept) 
   3.519429 

Scale= 0.7386973 

Loglik(model)= -41.7   Loglik(intercept only)= -41.7
n= 21 
\end{verbatim}

\hypertarget{estimation-de-la-loi-de-x-par-une-loi-exponentielle}{%
\subsubsection{Estimation de la loi de X par une loi exponentielle
:}\label{estimation-de-la-loi-de-x-par-une-loi-exponentielle}}

\begin{Shaded}
\begin{Highlighting}[]
\NormalTok{theta }\OtherTok{=} \FunctionTok{sum}\NormalTok{(finGMP) }\SpecialCharTok{/} \FunctionTok{sum}\NormalTok{(tempsGMP)}
\NormalTok{theta }
\end{Highlighting}
\end{Shaded}

\begin{verbatim}
[1] 0.02506964
\end{verbatim}

\begin{Shaded}
\begin{Highlighting}[]
\NormalTok{survexp }\OtherTok{=} \FunctionTok{survreg}\NormalTok{(donnF }\SpecialCharTok{\textasciitilde{}} \DecValTok{1}\NormalTok{, }\AttributeTok{dist =} \StringTok{"exponential"}\NormalTok{)}
\NormalTok{lambda }\OtherTok{=} \FunctionTok{exp}\NormalTok{(}\SpecialCharTok{{-}}\NormalTok{survexp}\SpecialCharTok{$}\NormalTok{coefficients)}
\NormalTok{lambda}
\end{Highlighting}
\end{Shaded}

\begin{verbatim}
(Intercept) 
 0.02506964 
\end{verbatim}

\hypertarget{examen-2018}{%
\section{Examen 2018 :}\label{examen-2018}}

\hypertarget{exercice-2}{%
\subsection{Exercice 2 :}\label{exercice-2}}

\hypertarget{importation-des-donnuxe9es-et-traitement-de-la-base}{%
\subsubsection{Importation des données et traitement de la base
:}\label{importation-des-donnuxe9es-et-traitement-de-la-base}}

\begin{Shaded}
\begin{Highlighting}[]
\FunctionTok{library}\NormalTok{(StMoMo)}
\NormalTok{d }\OtherTok{=}\NormalTok{ EWMaleData}
\NormalTok{De }\OtherTok{=}\NormalTok{ d}\SpecialCharTok{$}\NormalTok{Dxt }\CommentTok{\# décès}

\NormalTok{ages }\OtherTok{=}\NormalTok{ d}\SpecialCharTok{$}\NormalTok{ages}
\NormalTok{annees }\OtherTok{=}\NormalTok{ d}\SpecialCharTok{$}\NormalTok{years}

\NormalTok{Ex }\OtherTok{=}\NormalTok{ EWMaleData}\SpecialCharTok{$}\NormalTok{Ext  }\CommentTok{\# Expositions en milieu d\textquotesingle{}années}
\NormalTok{Lx }\OtherTok{=}\NormalTok{ Ex }\SpecialCharTok{+}\NormalTok{ De }\SpecialCharTok{/} \DecValTok{2} \CommentTok{\# Exposition en début d\textquotesingle{}année (approximation)}

\CommentTok{\# Calcul des taux de mortalité bruts pour 2011 : }
\NormalTok{q }\OtherTok{=}\NormalTok{ De[, }\StringTok{"2011"}\NormalTok{] }\SpecialCharTok{/}\NormalTok{ Lx[, }\StringTok{"2011"}\NormalTok{] }\CommentTok{\# taux bruts}

\FunctionTok{plot}\NormalTok{(}
\NormalTok{  ages,}
\NormalTok{  q,}
  \AttributeTok{type =} \StringTok{\textquotesingle{}l\textquotesingle{}}\NormalTok{,}
  \AttributeTok{main =} \StringTok{"Taux brut de mortalité"}\NormalTok{,}
  \AttributeTok{col =}\NormalTok{ palette\_couleur[}\DecValTok{1}\NormalTok{],}
  \AttributeTok{lwd =} \DecValTok{3}
\NormalTok{)}
\end{Highlighting}
\end{Shaded}

\begin{center}\includegraphics[width=0.95\linewidth]{1-MEMO_MODELE_DUREE_files/figure-latex/unnamed-chunk-19-1} \end{center}

\begin{Shaded}
\begin{Highlighting}[]
\FunctionTok{plot}\NormalTok{(}
\NormalTok{  ages,}
  \FunctionTok{log}\NormalTok{(q),}
  \AttributeTok{type =} \StringTok{\textquotesingle{}l\textquotesingle{}}\NormalTok{,}
  \AttributeTok{main =} \StringTok{"Logarithme des taux bruts de mortalité"}\NormalTok{,}
  \AttributeTok{col =}\NormalTok{ palette\_couleur[}\DecValTok{1}\NormalTok{],}
  \AttributeTok{lwd =} \DecValTok{3}
\NormalTok{)}
\end{Highlighting}
\end{Shaded}

\begin{center}\includegraphics[width=0.95\linewidth]{1-MEMO_MODELE_DUREE_files/figure-latex/unnamed-chunk-19-2} \end{center}

\hypertarget{calibration-dun-moduxe8le-de-makeham-gompertz}{%
\subsubsection{Calibration d'un modèle de Makeham-Gompertz
:}\label{calibration-dun-moduxe8le-de-makeham-gompertz}}

\hypertarget{utilisation-du-package-fmsb}{%
\paragraph{Utilisation du package fmsb
:}\label{utilisation-du-package-fmsb}}

Utilisation de la fonction fitGm pour calibrer le modèle
\(h(x) = C + A \times exp(\beta_x)\)

Avec la fonction fitGm on peut faire le lien avec l'autre paramétrage du
type :

\(h(x) = \alpha + \beta \times \gamma^{x}\) où \(x\) représente l'âge.

\begin{Shaded}
\begin{Highlighting}[]
\FunctionTok{library}\NormalTok{(fmsb)}
\NormalTok{fit }\OtherTok{=} \FunctionTok{fitGM}\NormalTok{(}\AttributeTok{data =}\NormalTok{ q) }

\NormalTok{A }\OtherTok{=}\NormalTok{ fit[}\DecValTok{1}\NormalTok{]}
\NormalTok{B }\OtherTok{=}\NormalTok{ fit[}\DecValTok{2}\NormalTok{]}
\NormalTok{C }\OtherTok{=}\NormalTok{ fit[}\DecValTok{3}\NormalTok{]}
\FunctionTok{cat}\NormalTok{(}\StringTok{"Modélisation fitGM :  }\SpecialCharTok{\textbackslash{}n}\StringTok{"}\NormalTok{)}
\end{Highlighting}
\end{Shaded}

\begin{verbatim}
Modélisation fitGM :  
\end{verbatim}

\begin{Shaded}
\begin{Highlighting}[]
\FunctionTok{c}\NormalTok{(A, B, C)}
\end{Highlighting}
\end{Shaded}

\begin{verbatim}
[1] 1.742762e-05 1.022779e-01 1.586628e-04
\end{verbatim}

\begin{Shaded}
\begin{Highlighting}[]
\CommentTok{\# Lien avec l\textquotesingle{}autre paramétrage :}
\NormalTok{alpha2 }\OtherTok{=}\NormalTok{ C}
\NormalTok{beta2 }\OtherTok{=}\NormalTok{ A}
\NormalTok{gamma2 }\OtherTok{=} \FunctionTok{exp}\NormalTok{(B)}
\FunctionTok{cat}\NormalTok{(}\StringTok{"Apha, Beta, Gamma : }\SpecialCharTok{\textbackslash{}n}\StringTok{"}\NormalTok{)}
\end{Highlighting}
\end{Shaded}

\begin{verbatim}
Apha, Beta, Gamma : 
\end{verbatim}

\begin{Shaded}
\begin{Highlighting}[]
\FunctionTok{c}\NormalTok{(alpha2, beta2, gamma2)  }
\end{Highlighting}
\end{Shaded}

\begin{verbatim}
[1] 1.586628e-04 1.742762e-05 1.107691e+00
\end{verbatim}

\begin{Shaded}
\begin{Highlighting}[]
\CommentTok{\# Construction du vecteur des probabilités de décès : }
\NormalTok{qM3 }\OtherTok{=} \DecValTok{1} \SpecialCharTok{{-}} \FunctionTok{exp}\NormalTok{(}\SpecialCharTok{{-}}\NormalTok{C) }\SpecialCharTok{*} \FunctionTok{exp}\NormalTok{(}\SpecialCharTok{{-}}\NormalTok{A }\SpecialCharTok{/}\NormalTok{ B }\SpecialCharTok{*} \FunctionTok{exp}\NormalTok{(B }\SpecialCharTok{*}\NormalTok{ ages) }\SpecialCharTok{*}\NormalTok{ (}\FunctionTok{exp}\NormalTok{(B) }\SpecialCharTok{{-}} \DecValTok{1}\NormalTok{))}

\CommentTok{\# Représentation graphique de l\textquotesingle{}âge des individus : }
\FunctionTok{plot}\NormalTok{(}
\NormalTok{  ages,}
\NormalTok{  q,}
  \AttributeTok{type =} \StringTok{\textquotesingle{}l\textquotesingle{}}\NormalTok{,}
  \AttributeTok{ylab =} \StringTok{"Probabilité de décès"}\NormalTok{,}
  \AttributeTok{xlab =} \StringTok{"Ages"}\NormalTok{,}
  \AttributeTok{main =} \StringTok{"Comparaison des taux de mortalités observés et estimés"}\NormalTok{,}
  \AttributeTok{col =}\NormalTok{ palette\_couleur[}\DecValTok{1}\NormalTok{],}
  \AttributeTok{lwd =} \DecValTok{2}
\NormalTok{)}
\FunctionTok{lines}\NormalTok{(ages, qM3, }\AttributeTok{col =}\NormalTok{ palette\_couleur[}\DecValTok{2}\NormalTok{], }\AttributeTok{lwd =} \DecValTok{2}\NormalTok{)}
\FunctionTok{legend}\NormalTok{(}
  \DecValTok{1}\NormalTok{,}
  \FloatTok{0.3}\NormalTok{,}
  \AttributeTok{lty =} \DecValTok{1}\NormalTok{,}
  \AttributeTok{cex =} \DecValTok{1}\NormalTok{,}
  \AttributeTok{legend =} \FunctionTok{c}\NormalTok{(}\StringTok{"q observés"}\NormalTok{, }\StringTok{"q estimés"}\NormalTok{),}
  \AttributeTok{col =}\NormalTok{ palette\_couleur[}\DecValTok{1}\SpecialCharTok{:}\DecValTok{2}\NormalTok{]}
\NormalTok{)}
\end{Highlighting}
\end{Shaded}

\begin{center}\includegraphics[width=0.95\linewidth]{1-MEMO_MODELE_DUREE_files/figure-latex/unnamed-chunk-20-1} \end{center}

\begin{Shaded}
\begin{Highlighting}[]
\CommentTok{\# Comparaison des taux de mortalités logarithmiques : }

\FunctionTok{plot}\NormalTok{(}
\NormalTok{  ages,}
  \FunctionTok{log}\NormalTok{(q),}
  \AttributeTok{type =} \StringTok{\textquotesingle{}l\textquotesingle{}}\NormalTok{,}
  \AttributeTok{ylab =} \StringTok{"Probabilité de décès"}\NormalTok{,}
  \AttributeTok{xlab =} \StringTok{"Ages"}\NormalTok{,}
  \AttributeTok{main =} \StringTok{"Comparaison des log de taux de mortalités observés et estimés"}\NormalTok{,}
  \AttributeTok{col =}\NormalTok{ palette\_couleur[}\DecValTok{1}\NormalTok{],}
  \AttributeTok{lwd =} \DecValTok{2}\NormalTok{)}
\FunctionTok{lines}\NormalTok{(ages, }\FunctionTok{log}\NormalTok{(qM3), }\AttributeTok{col =}\NormalTok{ palette\_couleur[}\DecValTok{2}\NormalTok{], }\AttributeTok{lwd =} \DecValTok{2}\NormalTok{)}
\FunctionTok{legend}\NormalTok{(}\StringTok{"topleft"}\NormalTok{,}
  \AttributeTok{lty =} \DecValTok{1}\NormalTok{,}
  \AttributeTok{cex =} \DecValTok{1}\NormalTok{,}
  \AttributeTok{legend =} \FunctionTok{c}\NormalTok{(}\StringTok{"q observés"}\NormalTok{, }\StringTok{"q G{-}M"}\NormalTok{),}
  \AttributeTok{col =}\NormalTok{ palette\_couleur[}\DecValTok{1}\SpecialCharTok{:}\DecValTok{2}\NormalTok{]}
\NormalTok{)}
\end{Highlighting}
\end{Shaded}

\begin{center}\includegraphics[width=0.95\linewidth]{1-MEMO_MODELE_DUREE_files/figure-latex/unnamed-chunk-20-2} \end{center}

Interprétation des résultats :

Le modèle de Gompertz - Makeham, avec h croissant, ne peut pas modéliser
correctement la mortalité aux âges inférieurs à 20 ans.

\hypertarget{utilisation-du-package-mortalitylaws}{%
\paragraph{Utilisation du package MortalityLaws
:}\label{utilisation-du-package-mortalitylaws}}

\begin{Shaded}
\begin{Highlighting}[]
\FunctionTok{library}\NormalTok{(MortalityLaws)}

\CommentTok{\#availableLaws() \# Liste des modèle de mortalité du package }

\NormalTok{fit }\OtherTok{=} \FunctionTok{MortalityLaw}\NormalTok{(}\AttributeTok{x =} \DecValTok{0}\SpecialCharTok{:}\DecValTok{100}\NormalTok{, }\AttributeTok{qx =}\NormalTok{ q, }\AttributeTok{law =} \StringTok{"makeham"}\NormalTok{)  }\CommentTok{\#modèle h(x)= C + A exp(Bx)}
\NormalTok{fit}\SpecialCharTok{$}\NormalTok{coefficients}
\end{Highlighting}
\end{Shaded}

\begin{verbatim}
           A            B            C 
0.0000251412 0.0953918954 0.0001246354 
\end{verbatim}

\begin{Shaded}
\begin{Highlighting}[]
\NormalTok{A }\OtherTok{=}\NormalTok{ fit}\SpecialCharTok{$}\NormalTok{coefficients[}\StringTok{"A"}\NormalTok{]}
\NormalTok{B }\OtherTok{=}\NormalTok{ fit}\SpecialCharTok{$}\NormalTok{coefficients[}\StringTok{"B"}\NormalTok{]}
\NormalTok{C }\OtherTok{=}\NormalTok{ fit}\SpecialCharTok{$}\NormalTok{coefficients[}\StringTok{"C"}\NormalTok{]}
\FunctionTok{c}\NormalTok{(A, B, C)}
\end{Highlighting}
\end{Shaded}

\begin{verbatim}
           A            B            C 
0.0000251412 0.0953918954 0.0001246354 
\end{verbatim}

\begin{Shaded}
\begin{Highlighting}[]
\CommentTok{\# Lien avec l\textquotesingle{}autre paramétrage (h(x)= alpha + beta gamma\^{}x)}
\NormalTok{alpha2 }\OtherTok{=}\NormalTok{ C}
\NormalTok{beta2 }\OtherTok{=}\NormalTok{ A}
\NormalTok{gamma2 }\OtherTok{=} \FunctionTok{exp}\NormalTok{(B)}
\FunctionTok{c}\NormalTok{(alpha2, beta2, gamma2)}
\end{Highlighting}
\end{Shaded}

\begin{verbatim}
           C            A            B 
0.0001246354 0.0000251412 1.1000898909 
\end{verbatim}

\begin{Shaded}
\begin{Highlighting}[]
\CommentTok{\# Estimation du taux de moralité de Lee{-}Carter }
\NormalTok{qM4 }\OtherTok{=} \DecValTok{1} \SpecialCharTok{{-}} \FunctionTok{exp}\NormalTok{(}\SpecialCharTok{{-}}\NormalTok{C) }\SpecialCharTok{*} \FunctionTok{exp}\NormalTok{(}\SpecialCharTok{{-}}\NormalTok{A }\SpecialCharTok{/}\NormalTok{ B }\SpecialCharTok{*} \FunctionTok{exp}\NormalTok{(B }\SpecialCharTok{*}\NormalTok{ ages) }\SpecialCharTok{*}\NormalTok{ (}\FunctionTok{exp}\NormalTok{(B) }\SpecialCharTok{{-}} \DecValTok{1}\NormalTok{))}


\CommentTok{\# Représentation graphique et comparaison : }

\FunctionTok{plot}\NormalTok{(}
\NormalTok{  ages,}
  \FunctionTok{log}\NormalTok{(q),}
  \AttributeTok{type =} \StringTok{\textquotesingle{}l\textquotesingle{}}\NormalTok{,}
  \AttributeTok{ylab =} \StringTok{"Probabilité de décès"}\NormalTok{,}
  \AttributeTok{xlab =} \StringTok{"Ages"}\NormalTok{,}
  \AttributeTok{main =} \StringTok{"Comparaison des log de taux de mortalités observés et estimés"}\NormalTok{,}
  \AttributeTok{col =}\NormalTok{ palette\_couleur[}\DecValTok{1}\NormalTok{],}
  \AttributeTok{lwd =} \DecValTok{2}\NormalTok{)}
\FunctionTok{lines}\NormalTok{(ages, }\FunctionTok{log}\NormalTok{(qM3), }\AttributeTok{col =}\NormalTok{ palette\_couleur[}\DecValTok{2}\NormalTok{], }\AttributeTok{lwd =} \DecValTok{2}\NormalTok{)}
\FunctionTok{lines}\NormalTok{(ages, }\FunctionTok{log}\NormalTok{(qM4), }\AttributeTok{col =}\NormalTok{ palette\_couleur[}\DecValTok{3}\NormalTok{], }\AttributeTok{lwd =} \DecValTok{2}\NormalTok{)}
\FunctionTok{legend}\NormalTok{(}\StringTok{"topleft"}\NormalTok{,}
  \AttributeTok{lty =} \DecValTok{1}\NormalTok{,}
  \AttributeTok{cex =} \DecValTok{1}\NormalTok{,}
  \AttributeTok{legend =} \FunctionTok{c}\NormalTok{(}\StringTok{"q observés"}\NormalTok{, }\StringTok{"q G{-}M"}\NormalTok{, }\StringTok{"q G{-}M bis"}\NormalTok{),}
  \AttributeTok{col =}\NormalTok{ palette\_couleur[}\DecValTok{1}\SpecialCharTok{:}\DecValTok{3}\NormalTok{]}
\NormalTok{)}
\end{Highlighting}
\end{Shaded}

\begin{center}\includegraphics[width=0.95\linewidth]{1-MEMO_MODELE_DUREE_files/figure-latex/unnamed-chunk-21-1} \end{center}

\hypertarget{moduxe9lisation-de-lee-carter}{%
\subsubsection{Modélisation de Lee Carter
:}\label{moduxe9lisation-de-lee-carter}}

Rappels sur la modélisation de Lee Carter :

\(ln(\mu{(x,t)}) = \alpha_x + \beta_x \times k_t + \epsilon_{(x,t)}\)

Avec : \alpha\_x = la valeur moyenne

\[
\left\{
\begin{array}{l}
\alpha_x : \text{la valeur moyenne} \\
k_t : \text{correspond à une évolution générale dans le temps}
\beta_x : \text{la sensibilité du taux instantané par rapport à une variation de k_t} \\
\end{array}
\right.
\]

\begin{Shaded}
\begin{Highlighting}[]
\FunctionTok{library}\NormalTok{(forecast)}
\FunctionTok{library}\NormalTok{(demography)}
\NormalTok{muh }\OtherTok{=}\NormalTok{ De }\SpecialCharTok{/}\NormalTok{ Ex}
\NormalTok{Baseh }\OtherTok{=} \FunctionTok{demogdata}\NormalTok{(}
  \AttributeTok{data =}\NormalTok{ muh,}
  \AttributeTok{pop =}\NormalTok{ Ex,}
  \AttributeTok{ages =}\NormalTok{ ages,}
  \AttributeTok{years =}\NormalTok{ annees,}
  \AttributeTok{type =} \StringTok{"mortality"}\NormalTok{,}
  \AttributeTok{label =} \StringTok{\textquotesingle{}G.B.\textquotesingle{}}\NormalTok{,}
  \AttributeTok{name =} \StringTok{\textquotesingle{}Hommes\textquotesingle{}}\NormalTok{,}
  \AttributeTok{lambda =} \DecValTok{1}\NormalTok{)}

\NormalTok{lch }\OtherTok{=} \FunctionTok{lca}\NormalTok{(Baseh) }\CommentTok{\# Lancement du modèle de Lee{-}Carter}

\CommentTok{\# Estimation de alpha\_x}
\FunctionTok{plot}\NormalTok{(lch}\SpecialCharTok{$}\NormalTok{age, lch}\SpecialCharTok{$}\NormalTok{ax, }\AttributeTok{col =} \StringTok{"blue"}\NormalTok{)}
\end{Highlighting}
\end{Shaded}

\begin{center}\includegraphics[width=0.95\linewidth]{1-MEMO_MODELE_DUREE_files/figure-latex/unnamed-chunk-22-1} \end{center}

\begin{Shaded}
\begin{Highlighting}[]
\CommentTok{\# Estimation de beta\_x}
\FunctionTok{plot}\NormalTok{(lch}\SpecialCharTok{$}\NormalTok{age, lch}\SpecialCharTok{$}\NormalTok{bx, }\AttributeTok{col =} \StringTok{"blue"}\NormalTok{)}
\end{Highlighting}
\end{Shaded}

\begin{center}\includegraphics[width=0.95\linewidth]{1-MEMO_MODELE_DUREE_files/figure-latex/unnamed-chunk-22-2} \end{center}

\begin{Shaded}
\begin{Highlighting}[]
\CommentTok{\# Estimation des k\_t}
\NormalTok{kt }\OtherTok{=}\NormalTok{ lch}\SpecialCharTok{$}\NormalTok{kt}
\FunctionTok{plot}\NormalTok{(annees, kt)}
\end{Highlighting}
\end{Shaded}

\begin{center}\includegraphics[width=0.95\linewidth]{1-MEMO_MODELE_DUREE_files/figure-latex/unnamed-chunk-22-3} \end{center}

\hypertarget{muxe9thode-de-lee-carter-1992-projection-des-kt}{%
\paragraph{Méthode de Lee-Carter 1992 : Projection des
Kt}\label{muxe9thode-de-lee-carter-1992-projection-des-kt}}

Rappel: les Kt représentent

Hypothèse : \(k_t = k_{t-1}+ d + e_t\)

\begin{Shaded}
\begin{Highlighting}[]
\CommentTok{\# Projection des Kt à l\textquotesingle{}aide du modèle initial : }
\FunctionTok{plot}\NormalTok{(lch)}
\end{Highlighting}
\end{Shaded}

\begin{center}\includegraphics[width=0.95\linewidth]{1-MEMO_MODELE_DUREE_files/figure-latex/unnamed-chunk-23-1} \end{center}

\begin{Shaded}
\begin{Highlighting}[]
\NormalTok{proj }\OtherTok{=} \FunctionTok{forecast}\NormalTok{(lch, }\AttributeTok{h =} \DecValTok{20}\NormalTok{)}
\FunctionTok{plot}\NormalTok{(proj, }\AttributeTok{plot.type =} \StringTok{"component"}\NormalTok{)}
\end{Highlighting}
\end{Shaded}

\begin{center}\includegraphics[width=0.95\linewidth]{1-MEMO_MODELE_DUREE_files/figure-latex/unnamed-chunk-23-2} \end{center}

\begin{Shaded}
\begin{Highlighting}[]
\CommentTok{\# Projection des Kt à l\textquotesingle{}aide du modèle ARIMA : }
\NormalTok{ar }\OtherTok{=} \FunctionTok{auto.arima}\NormalTok{(kt)}
\FunctionTok{plot}\NormalTok{(}\FunctionTok{forecast}\NormalTok{(ar, }\AttributeTok{h =} \DecValTok{20}\NormalTok{))}
\end{Highlighting}
\end{Shaded}

\begin{center}\includegraphics[width=0.95\linewidth]{1-MEMO_MODELE_DUREE_files/figure-latex/unnamed-chunk-23-3} \end{center}

\hypertarget{moduxe8le-de-lee-carter-sans-ajustement-des-kt}{%
\paragraph{Modèle de Lee Carter sans ajustement des Kt
:}\label{moduxe8le-de-lee-carter-sans-ajustement-des-kt}}

\begin{Shaded}
\begin{Highlighting}[]
\CommentTok{\# Mo}
\DocumentationTok{\#\# L.C. sans ajustement des k\_t}
\NormalTok{lch\_sans }\OtherTok{=} \FunctionTok{lca}\NormalTok{(Baseh, }\AttributeTok{adjust =} \StringTok{"none"}\NormalTok{)}
\FunctionTok{plot}\NormalTok{(lch}\SpecialCharTok{$}\NormalTok{year,}
\NormalTok{     lch}\SpecialCharTok{$}\NormalTok{kt,}
     \AttributeTok{col =} \StringTok{"blue"}\NormalTok{,}
     \AttributeTok{type =} \StringTok{\textquotesingle{}l\textquotesingle{}}\NormalTok{,}
     \AttributeTok{main =} \StringTok{"Effet ajustement sur les k\_t"}\NormalTok{)}
\FunctionTok{lines}\NormalTok{(lch\_sans}\SpecialCharTok{$}\NormalTok{year, lch\_sans}\SpecialCharTok{$}\NormalTok{kt, }\AttributeTok{col =} \StringTok{\textquotesingle{}red\textquotesingle{}}\NormalTok{)}
\FunctionTok{legend}\NormalTok{(}
  \DecValTok{1960}\NormalTok{,}
  \SpecialCharTok{{-}}\DecValTok{20}\NormalTok{,}
  \AttributeTok{legend =} \FunctionTok{c}\NormalTok{(}\StringTok{"avec ajust."}\NormalTok{, }\StringTok{"sans ajust."}\NormalTok{),}
  \AttributeTok{col =} \FunctionTok{c}\NormalTok{(}\StringTok{"blue"}\NormalTok{, }\StringTok{"red"}\NormalTok{),}
  \AttributeTok{lty =} \DecValTok{1}\NormalTok{,}
  \AttributeTok{cex =} \FloatTok{0.8}
\NormalTok{)}
\end{Highlighting}
\end{Shaded}

\begin{center}\includegraphics[width=0.95\linewidth]{1-MEMO_MODELE_DUREE_files/figure-latex/unnamed-chunk-24-1} \end{center}

\hypertarget{comparaison-des-ruxe9sutlats}{%
\paragraph{Comparaison des résutlats
:}\label{comparaison-des-ruxe9sutlats}}

\begin{Shaded}
\begin{Highlighting}[]
\CommentTok{\# comparaison avec G.M.}
\NormalTok{predh }\OtherTok{=}\NormalTok{ lch}\SpecialCharTok{$}\NormalTok{fitted}\SpecialCharTok{$}\NormalTok{y  }\CommentTok{\# c\textquotesingle{}est log(mu\_\{x,t\}) qui est prédit}
\NormalTok{mupred2011 }\OtherTok{=} \FunctionTok{exp}\NormalTok{(predh[, }\DecValTok{51}\NormalTok{])}

\FunctionTok{plot}\NormalTok{(ages,q,}\AttributeTok{type=}\StringTok{\textquotesingle{}l\textquotesingle{}}\NormalTok{)}
\FunctionTok{lines}\NormalTok{(ages, qM3,}\AttributeTok{col=}\StringTok{\textquotesingle{}blue\textquotesingle{}}\NormalTok{) }\CommentTok{\# G.M.}
\FunctionTok{lines}\NormalTok{(ages,mupred2011,}\AttributeTok{col=}\StringTok{\textquotesingle{}red\textquotesingle{}}\NormalTok{)}
\end{Highlighting}
\end{Shaded}

\begin{center}\includegraphics[width=0.95\linewidth]{1-MEMO_MODELE_DUREE_files/figure-latex/unnamed-chunk-25-1} \end{center}

\begin{Shaded}
\begin{Highlighting}[]
\FunctionTok{plot}\NormalTok{(mupred2011}\SpecialCharTok{{-}}\NormalTok{qM3)}
\end{Highlighting}
\end{Shaded}

\begin{center}\includegraphics[width=0.95\linewidth]{1-MEMO_MODELE_DUREE_files/figure-latex/unnamed-chunk-25-2} \end{center}

\begin{Shaded}
\begin{Highlighting}[]
\FunctionTok{max}\NormalTok{(}\FunctionTok{abs}\NormalTok{(mupred2011}\SpecialCharTok{{-}}\NormalTok{qM3))}
\end{Highlighting}
\end{Shaded}

\begin{verbatim}
[1] 0.06453279
\end{verbatim}

\begin{Shaded}
\begin{Highlighting}[]
\CommentTok{\# comparaison graphique des modèles pour 2011}
\FunctionTok{plot}\NormalTok{(ages,}\FunctionTok{log}\NormalTok{(q))}
\FunctionTok{lines}\NormalTok{(ages,}\FunctionTok{log}\NormalTok{(qM3),}\AttributeTok{col=}\StringTok{\textquotesingle{}blue\textquotesingle{}}\NormalTok{) }\CommentTok{\# GM (fmsb)}
\FunctionTok{lines}\NormalTok{(ages,}\FunctionTok{log}\NormalTok{(qM4),}\AttributeTok{col=}\StringTok{\textquotesingle{}green\textquotesingle{}}\NormalTok{) }\CommentTok{\# mortalitylaw}
\FunctionTok{lines}\NormalTok{(ages,predh[,}\DecValTok{51}\NormalTok{],}\AttributeTok{col=}\StringTok{\textquotesingle{}red\textquotesingle{}}\NormalTok{) }\CommentTok{\# Lee Carter}
\end{Highlighting}
\end{Shaded}

\begin{center}\includegraphics[width=0.95\linewidth]{1-MEMO_MODELE_DUREE_files/figure-latex/unnamed-chunk-25-3} \end{center}

\hypertarget{calcul-des-rentes}{%
\subsubsection{Calcul des rentes :}\label{calcul-des-rentes}}

\begin{Shaded}
\begin{Highlighting}[]
\DocumentationTok{\#\#\# calcul des rentes}
\CommentTok{\# Projections des \textbackslash{}mu\{x,t\} dans le futur :}
\CommentTok{\# projection standard du modèle de Lee{-}Carter :}
\NormalTok{projh}\OtherTok{=}\FunctionTok{forecast}\NormalTok{(lch,}\AttributeTok{h=}\DecValTok{70}\NormalTok{)}\SpecialCharTok{$}\NormalTok{rate}\SpecialCharTok{$}\NormalTok{Hommes}

\FunctionTok{dim}\NormalTok{(projh)}
\end{Highlighting}
\end{Shaded}

\begin{verbatim}
[1] 101  70
\end{verbatim}

\begin{Shaded}
\begin{Highlighting}[]
\FunctionTok{colnames}\NormalTok{(projh) }\OtherTok{=} \DecValTok{2012}\SpecialCharTok{:}\NormalTok{(}\DecValTok{2012} \SpecialCharTok{+} \DecValTok{69}\NormalTok{)}
\FunctionTok{rownames}\NormalTok{(projh) }\OtherTok{=} \DecValTok{0}\SpecialCharTok{:}\DecValTok{100}
\FunctionTok{View}\NormalTok{(projh)}

\CommentTok{\# nous souhaitons calculer la prime pure d\textquotesingle{}une rente viagère }
\CommentTok{\# à partir de 2012 pour l\textquotesingle{}âge de 65 ans}
\CommentTok{\# a\_x(t)= \textbackslash{}sum\_\{k\textbackslash{}ge 0\} \{ \textbackslash{}prod\_\{j=0\}\^{}k exp({-}\textbackslash{}mu\_\{x+j\}(t+j)) *1/(1+r)\^{}(k+1) \}}

\NormalTok{r }\OtherTok{=} \FloatTok{0.035} \CommentTok{\# valeur du taux choisi pour le facteur d\textquotesingle{}actualisation}

\CommentTok{\# calcul de a\_65(2012) pour les hommes :}

\NormalTok{L }\OtherTok{=} \FunctionTok{length}\NormalTok{(}\DecValTok{66}\SpecialCharTok{:}\DecValTok{101}\NormalTok{)}
\NormalTok{mu }\OtherTok{=}\NormalTok{ projh[}\DecValTok{66}\SpecialCharTok{:}\DecValTok{101}\NormalTok{, }\DecValTok{1}\SpecialCharTok{:}\NormalTok{L] }\CommentTok{\# on limite aux âges 65{-}100}
\NormalTok{dmu }\OtherTok{=} \FunctionTok{diag}\NormalTok{(mu)}
\NormalTok{prodexpmu }\OtherTok{=} \FunctionTok{cumprod}\NormalTok{(}\FunctionTok{exp}\NormalTok{(}\SpecialCharTok{{-}}\NormalTok{dmu))}
\NormalTok{a }\OtherTok{=} \DecValTok{0}
\ControlFlowTok{for}\NormalTok{ (k }\ControlFlowTok{in} \DecValTok{1}\SpecialCharTok{:}\FunctionTok{length}\NormalTok{(dmu))}
\NormalTok{\{}
\NormalTok{  a }\OtherTok{=}\NormalTok{ a }\SpecialCharTok{+} \DecValTok{1} \SpecialCharTok{/}\NormalTok{ (}\DecValTok{1} \SpecialCharTok{+}\NormalTok{ r) }\SpecialCharTok{\^{}}\NormalTok{ (k) }\SpecialCharTok{*}\NormalTok{ prodexpmu[k]}
\NormalTok{\}}
\NormalTok{a  }\CommentTok{\# 13.164}
\end{Highlighting}
\end{Shaded}

\begin{verbatim}
[1] 13.16419
\end{verbatim}

\begin{Shaded}
\begin{Highlighting}[]
\CommentTok{\# Remarque : si on prolonge jusqu\textquotesingle{}à 120 ans avec les mêmes \textbackslash{}mu(x,t) ?}
\CommentTok{\# (pour vérifier si négliger les âges \textgreater{} 110 est justifié)}
\NormalTok{dmu120 }\OtherTok{=} \FunctionTok{c}\NormalTok{(dmu, }\FunctionTok{rep}\NormalTok{(dmu[L], }\DecValTok{20}\NormalTok{))}
\NormalTok{prodexpmu120 }\OtherTok{=} \FunctionTok{cumprod}\NormalTok{(}\FunctionTok{exp}\NormalTok{(}\SpecialCharTok{{-}}\NormalTok{dmu120))}
\NormalTok{a120 }\OtherTok{=} \DecValTok{0}
\ControlFlowTok{for}\NormalTok{ (k }\ControlFlowTok{in} \DecValTok{1}\SpecialCharTok{:}\NormalTok{(L }\SpecialCharTok{+} \DecValTok{20}\NormalTok{))}
\NormalTok{\{}
\NormalTok{  a120 }\OtherTok{=}\NormalTok{ a120 }\SpecialCharTok{+} \DecValTok{1} \SpecialCharTok{/}\NormalTok{ (}\DecValTok{1} \SpecialCharTok{+}\NormalTok{ r) }\SpecialCharTok{\^{}}\NormalTok{ (k) }\SpecialCharTok{*}\NormalTok{ prodexpmu120[k]}
\NormalTok{\}}
\NormalTok{a120 }\CommentTok{\# 13.174}
\end{Highlighting}
\end{Shaded}

\begin{verbatim}
[1] 13.17422
\end{verbatim}

\begin{Shaded}
\begin{Highlighting}[]
\CommentTok{\# Comparaison avec G.M. I  (fmsb)}
\NormalTok{dmu }\OtherTok{=}\NormalTok{ qM3[}\DecValTok{66}\SpecialCharTok{:}\DecValTok{101}\NormalTok{]}
\NormalTok{prodexpmu }\OtherTok{=} \FunctionTok{cumprod}\NormalTok{(}\FunctionTok{exp}\NormalTok{(}\SpecialCharTok{{-}}\NormalTok{dmu))}
\NormalTok{a }\OtherTok{=} \DecValTok{0}
\ControlFlowTok{for}\NormalTok{ (k }\ControlFlowTok{in} \DecValTok{1}\SpecialCharTok{:}\FunctionTok{length}\NormalTok{(dmu))}
\NormalTok{\{}
\NormalTok{  a }\OtherTok{=}\NormalTok{ a }\SpecialCharTok{+} \DecValTok{1} \SpecialCharTok{/}\NormalTok{ (}\DecValTok{1} \SpecialCharTok{+}\NormalTok{ r) }\SpecialCharTok{\^{}}\NormalTok{ (k) }\SpecialCharTok{*}\NormalTok{ prodexpmu[k]}
\NormalTok{\}}
\NormalTok{a}
\end{Highlighting}
\end{Shaded}

\begin{verbatim}
[1] 12.1337
\end{verbatim}

\begin{Shaded}
\begin{Highlighting}[]
\CommentTok{\# 12.13}

\CommentTok{\# Comparaison avec G.M. II  (Mortalitylaw)}
\NormalTok{dmu }\OtherTok{=}\NormalTok{ qM4[}\DecValTok{66}\SpecialCharTok{:}\DecValTok{101}\NormalTok{]}
\NormalTok{prodexpmu }\OtherTok{=} \FunctionTok{cumprod}\NormalTok{(}\FunctionTok{exp}\NormalTok{(}\SpecialCharTok{{-}}\NormalTok{dmu))}
\NormalTok{a }\OtherTok{=} \DecValTok{0}
\ControlFlowTok{for}\NormalTok{ (k }\ControlFlowTok{in} \DecValTok{1}\SpecialCharTok{:}\FunctionTok{length}\NormalTok{(dmu))}
\NormalTok{\{}
\NormalTok{  a }\OtherTok{=}\NormalTok{ a }\SpecialCharTok{+} \DecValTok{1} \SpecialCharTok{/}\NormalTok{ (}\DecValTok{1} \SpecialCharTok{+}\NormalTok{ r) }\SpecialCharTok{\^{}}\NormalTok{ (k) }\SpecialCharTok{*}\NormalTok{ prodexpmu[k]}
\NormalTok{\}}
\NormalTok{a}
\end{Highlighting}
\end{Shaded}

\begin{verbatim}
[1] 12.77115
\end{verbatim}

\begin{Shaded}
\begin{Highlighting}[]
\CommentTok{\# 12.77}
\end{Highlighting}
\end{Shaded}

\hypertarget{examen-2019}{%
\section{Examen 2019 :}\label{examen-2019}}

\hypertarget{examen-2023-2024}{%
\section{Examen 2023-2024 :}\label{examen-2023-2024}}

\end{document}
